\documentclass[a4paper,12pt,oneside,titlepage]{article}
\usepackage[brazilian,english]{babel}
\usepackage{tikz}
\usepackage[utf8]{inputenc}
\usepackage{graphicx,color}
\usepackage{makeidx}
\usepackage{float}
\usepackage{listings}
\makeindex
%%%%%%%%%%%%%%%%%%%%%%%%%%%%%%%%%%%%%%%%%%%%%%%%%%%%%%%%%%
\usepackage{indentfirst}
\usepackage{url}
% pacotes para incrementar recursos matematicos
% amsmath -> recursos avancados de matematica
% amssymb -> symbolos e fontes adicionais (ele inclui amsfonts)
% amsthm -> teorema de demonstracao
\usepackage{amsmath,amssymb} 
\usepackage{amsthm} 

%%%%%%%%%%%%%%%%%%%%%%%%%%%%%%%%%%%%%%%%%%%%%%%
% para configurar a lista enumerada
\usepackage{enumerate}

%%%%%%%%%%%%%%%%%%%%%%%%%%%%%%%%%%%%%%%%%%%%%%%
% tabela longa que quebra entre p\'aginas
\usepackage{longtable}

% linhas duplas na tabela
\usepackage{hhline}

%%%%%%%%%%%%%%%%%%%%%%%%%%%%%%%%%%%%%%%%%%%%%%%%%
% Para incluir imagens externas no LaTeX
% caixa gr\'afica
\usepackage{graphicx}

%%%%%%%%%%%%%%%%%%%%%%%%%%%%%%%%%%%%%%
% Pacotes graficos para figuras
% usando comandos de LaTeX

% Ative o pict2e, se for usar ambiente picture
%  \usepackage{pict2e}



%%%%%%%%%%%%%%%%%%%%%%%%%%%%%%%%%%%%%%%%%%%%%%%%%%%%%%%%%%%%%%%%%%
% Caso queira indentar a primeira linha do capitulo/secao, ative-o
% \usepackage{indentfirst}

%%%%%%%%%%%%%%%%%%%%
% indice remissivo
%\usepackage{makeidx}
%\makeindex % ativar (necess\'ario)

%%%%%%%%%%%%%%%%%%%%%%%%%%%%%%%%%%%%%%%%%%%%%%%%%%%%%%%%%%%%%%%%%%%%
% para acertar margens
% No caso de PCTeX 4.0 ou anterior, os arquivos 
% geometry.sty e geometry.cfg devem ficar junto com 
% o arquivo tex em edicao por n\~ao fazer parte da instala\c{c}\~ao.

\usepackage{geometry}
% Acerto de margens:
% lmargin -> left (esquerda). Interna, se frente/verso
% rmargin -> right (direita). Externa, se frente/verso
% tmargin -> top (superior).
% bmargin -> bot (inferior).
\geometry{lmargin=3.0cm,rmargin=2.0cm,tmargin=2.5cm,bmargin=2.0cm}

\usepackage{cite}

\begin{document}

\begin{center}
	

\textbf{Análise da Flexibilização do Isolamento Social do Ceará: Média Móvel e  Modelo SIR}


\end{center}
\begin{flushright}
	\textbf{Vinícius Barbosa Godinho\footnote{E-mail: viniciusbgodinho@gmail.com ,   twitter:@vbgodinho.}} \\
	\textbf{14/07/2020}
\end{flushright}


\section{Introdução}	


O estado cearense é um dos mais afetados pelo coronavírus, sendo o terceiro com maior número de óbitos, ficando atrás apenas de São Paulo e Rio de Janeiro, primeiro e segundo, respectivamente. Porém o Ceará tem apresentando uma queda da média de novos óbitos desde o mês de junho e adotando uma flexibilização do isolamento. Assim, é importante entender os motivos dessa queda e analisar o plano de flexibilização do governo que é dividido em uma fase de transição e outras quatro fases de abertura.

O intuito desse estudo é demonstrar a trajetória dos novos óbitos dos municípios mais afetados e estimar os números de infectados para 10 dias utilizando um modelo SIR\footnote{Dados da Secretaria Estadual de Saúde}. Dos municípios analisados Fortaleza se encontra na fase 3, a região metropolitana na fase 2, a região do Cariri\footnote{Foi utilizado apenas Crato e Juazeiro do Norte} na fase transição e Sobral no isolamento rígido. 

  

\section{Análise Média Móvel: 7 dias}

Para analisar a média móvel foi utilizada uma média centrada para 7 dias. Para entender a trajetória de queda dos novos óbitos no estado é necessário destacar Fortaleza dos demais municípios, pois é justamente a maior queda na capital que puxou a queda no estado como podemos observar na figura \ref{g1}. 
	
	\begin{figure}[H]
		\centering
		\caption{Fortaleza}
		\input{tikz_fortal}
		\label{g1}
		\ \footnotesize Dados: SES. Disponibilizados: https://data.brasil.io/. Elaborado: Vinícius Godinho. 
		
	\end{figure}
	

O período destacado entre a linha tracejada foi o de isolamento social rígido. Pode-se perceber o efeito positivo gerado por esse isolamento, pois foi após o período de 3 semanas do decreto que iniciou a queda da média diária dos novos óbitos.

A figura \ref{g2} ilustra os municípios mais afetados, exceto Fortaleza. É possível observar que os municípios da região metropolitana mais afetados, Caucaia e Maracanaú, também apresentaram uma queda da média, porém a partir da primeira semana de junho. Já a região do Cariri registrou um aumento médio do número de casos, mostrando uma interiorização do vírus muito preocupante, visto a falta de infraestrutura do interior frente à capital, movimento que vem ocorrendo nos demais estados.	
	
	\begin{figure}[H]
		\centering
		\caption{Municípios Ceará}
		\input{tikz_cidades_ce}
		\label{g2}
		\ \footnotesize Dados: SES. Disponibilizados: https://data.brasil.io/. Elaborado: Vinícius Godinho. 
		
	\end{figure}
	

\section{Estimando o número de infectados para 10 dias - modelo SIR}

Para projetar o número de infectados foi utilizado o modelo epidemiológico SIR (Suscetíveis - Infectados. - Removidos) desenvolvido por Kermack e Mckendrick, apresentado na equação:

\begin{center}
	

$d_S/d_t=-(\beta*I*S)/N$ \\
$d_I/d_t=(\beta*I*S)N  - \gamma *I$ \\
$d_R/d_t=\gamma*I$ \\

Desse modo a taxa de transmissão do modelo é R0 \\

$R0=(\beta/\gamma)/N$


\end{center}


Para: S= suscetíveis, I= infectados, R = recuperados, N = população, R0= taxa de transmissão, $\gamma$= taxa de recuperação e $\beta$ = taxa de infecção.


Ou seja: para $R0 < 1$ o número de infectados decresce. Foi utilizado um período de infecção igual a 5,2 dias, desse modo a taxa de recuperação $(\gamma=1/5,2)$, como utilizado no modelo adotado pela UFPEL\footnote{ \url{https://wp.ufpel.edu.br/fentransporte/2020/04/09/a-evolucao-epidemica-do-covid-19-modelo-sir/}}. 


Desse modo, utilizando a taxa de transmissão disponibilizada pelo governo do Ceará\footnote{\url{https://indicadores.integrasus.saude.ce.gov.br/indicadores/indicadores-coronavirus/repro-efetiva-rt}}, podemos calcular a taxa de infecção $(\beta)$ e projetar o número de infectados para os próximos 10 dias, considerando as taxas a seguir constante: 


\begin{table}[H]
	\centering
	\begin{tabular}{rr}
		\hline
		 Região & R0 \\ 
		\hline
		Cariri & 0,95 \\
		Fortaleza & 0,81 \\
		Sobral & 0,98 \\
		\hline
	\end{tabular}
\end{table}	

\textbf{Previsão:}


\begin{table}[H]
	\centering
	\begin{tabular}{cccccccccc}
		&  &  Fortaleza&  &  & Cariri&  &  & Sobral &\\
		\hline
		Dias&S&I& R & S & I& R & S & I & R\\
		\hline
		 0 & 2606858 & 9528 & 28614 & 342772 & 1604 & 4741 & 138912 & 1029 & 7412 \\  
		 1 & 2605423 & 9165 & 30411 & 342486 & 1583 & 5047 & 138730 & 1013 & 7608 \\ 
		 2 & 2604044 & 8815 & 32139 & 342204 & 1562 & 5349 & 138552 & 998 & 7801 \\  
		 3 & 2602719 & 8478 & 33802 & 341926 & 1541 & 5648 & 138376 & 983 & 7992 \\  
		 4 & 2601445 & 8153 & 35401 & 341652 & 1521 & 5943 & 138203 & 968 & 8180 \\  
		 5 & 2600220 & 7840 & 36939 & 341382 & 1500 & 6233 & 138034 & 953 & 8365 \\  
		 6 & 2599043 & 7539 & 38417 & 341116 & 1480 & 6520 & 137867 & 938 & 8547 \\   
		 7 & 2597911 & 7248 & 39839 & 340854 & 1459 & 6802 & 137703 & 923 & 8726 \\  
		 8 & 2596824 & 6968 & 41206 & 340595 & 1439 & 7081 & 137541 & 908 & 8902 \\   
		 9 & 2595779 & 6699 & 42520 & 340340 & 1419 & 7356 & 137383 & 893 & 9075 \\   
		 10 & 2594775 & 6440 & 43783 & 340090 & 1399 & 7627 & 137227 & 879 & 9246 \\   
		\hline
	\end{tabular}
\end{table}
	

		
	
Podemos observar na figura \ref{g3} que se  a taxa de transmissão permanecer constante Fortaleza decrescerá o número de infectados, com uma inclinação da curva bem maior do que as outras regiões. Essas vão apresentando uma pequena queda, que só será continuada mantendo a $R0 < 1$. Assim deve-se ter bastante atenção com as etapas de flexibilização do Cariri e Sobral, pois essas regiões  apresentaram $R0 < 1$ apenas nesses últimos dias podendo voltar a ficar maior do que um.   	
	
	\begin{figure}[H]
		\centering
		\caption{Projeção Número de Infectados}
		\input{tikz_sir}
		\label{g3}
		\ \footnotesize Dados: SES-CE. Elaborado: Vinícius Godinho. 
		
	\end{figure}
	
	
\section{Conclusão}
	
A média dos novos óbitos do estado do Ceará caiu de forma considerável, puxada pela queda na capital. Dessa forma, podemos perceber  que o isolamento rígido em Fortaleza foi o ponto fundamental para a queda da média de novos óbitos.

Com essa taxa de transmissão constante o número de infectados em Fortaleza tende a continuar caindo, porém exige atenção e constante monitoramento visto que o resultado da flexibilização e reabertura pode refletir daqui algumas semanas uma segunda onda.

A interiorização do contágio é o ponto mais preocupante para o estado. A região do Cariri apesar de ter diminuído ao longo dos dias sua taxa de transmissão e alcançado $R0 < 1$, essa ainda é muito próxima de 1. Mesmo com a média móvel dos óbitos aumentando a região já se encontra na fase de transição do plano de flexibilização do estado. Desse modo, essa região deveria voltar para o isolamento rígido, assim como permanece Sobral. 
	
	
\end{document}	